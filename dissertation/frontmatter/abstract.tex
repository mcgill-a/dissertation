%!TEX root = ../dissertation.tex
% the abstract

% What is the problem?
Neural Machine Translation typically requires a huge volume of parallel data for the translation quality to be sufficient. Low-resource neural machine translation approaches aim to alleviate the necessity for an abundance of data.
% What was done?
Through a comprehensive review of the literature, this project analyses the most prevalent low-resource neural machine translation approaches. A research gap regarding the application of neural machine translation for the low-resource language Scottish Gaelic was identified, leading to the focus of back-translation data augmentation and transfer learning techniques to improve the translation quality for Scottish Gaelic. The trivial transfer learning and hierarchical transfer learning techniques identified were compared against a GRU baseline model.

% What was discovered?
It was discovered that when based on an analysis of the data corpus, restrictions to the sentence length and vocabulary size can lead to improvements in the quality of Scottish Gaelic translation. Furthermore, it was found that despite scoring lower by automatic translation evaluation metrics, translations from transfer learning models are more varied and relevant than a baseline model.
% What do the findings mean?
The findings demonstrate the significance of enforcing restrictions on the vocabulary size and sentence length of the data corpus. An analysis of these findings concludes with a suggestion that given a sufficient quantity of the data in the parent language, transfer learning may prove beneficial to the quality of Scottish Gaelic neural machine translation.