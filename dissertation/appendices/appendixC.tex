\chapter{Diary Sheets}
\label{AppendixC}

\subsubsection{Supervisor Meeting 1 - 28/09/2019}
This week we discussed the initial project overview that I had prepared beforehand. Dimitra gave me some alterations to clarify some minor details and provide more information sources for the context of the project.
%We also spoke about the upcoming deadlines such as the week 9 report and what will be included in it.
Dimitra suggested that for next week I should create two tables of academic references. One for papers relating to machine translation of Scottish Gaelic and the other for papers about machine translation for other low resource languages. Including the approaches and results on the tables should help identify methods that can be applied to Scottish Gaelic.


\subsubsection{Supervisor Meeting 2 - 04/10/2019}
We spoke about the reference tables I created for previous low resource machine translation research. The tables revealed that there is no published work on neural machine translation that has focuses on Scottish Gaelic. Most of the papers for low resource languages use various types of transfer learning so we decided that transfer learning should be the focus of the experiments.

Dimitra suggested that for next week I should aim to have made some progress on the introduction and plan out the structure of the dissertation (sections, headings and sub-headings). We also discussed more information sources that could be used for gathering training data such as the Scottish parliament website so I will look into these by the next meeting as well.


\subsubsection{Supervisor Meeting 3 - 11/10/2019}
We went over and made a few changes to the dissertation contents page structure I that created which includes the different sections I expect to cover in the dissertation. We also went over my dissertation introduction section and Dimitra gave a lot of helpful feedback which mainly involved adding more examples and explanations for various statements to give more context. For example, for the problem statement part I need to explain the importance of the issue and clarify what impact solving the problem could result in.

For next week I will make the changes to my introduction and ideally have made some good progress on the first draft of my literature review. 

\subsubsection{Supervisor Meeting 4 - 18/10/2019}
Over the past week I had made a start on the literature review, working on the low-resource translation approaches section. Dimitra read over it and gave me some feedback such as to add more diagrams, avoid direct quotations, and focus on explaining what the references found rather than criticising their findings. I also need to mention any figures in the text to bring the reader’s attention to them.

I also added some more information to the introduction as Dimitra recommended during our last meeting. The extra information gives more context to the project by clarifying why low-resource languages achieve poor results for neural machine translation. 

For next week we agreed that I should work on the Gaelic section of the literature review and get started on the machine translation section.

\subsubsection{Supervisor Meeting 5 - 25/10/2019}
This week I have been working on the literature review. I completed a section on data augmentation and added a section on Scottish Gaelic machine translation research. Previously I was planning to get started on the machine learning section once I had finished the Gaelic but during the week, we decided it would be better to focus on data augmentation first.

During the meeting Dimitra read over it and said it was good. She didn’t have any suggestions for changes to the work, just to keep up doing what I’ve been doing. 

We agreed that next week I will finish the sentence segmentation section and then focus on the machine translation section of the literature review. This involves writing about training data, translation techniques, and translation evaluation.
We also spoke the possibility of using a cloud platform service provider for the neural model training. This would be instead of using my own PC, which may take too long to train quite a few different models. I will look into this further in the next few weeks to see whether or not it would be worth it.

\subsubsection{Supervisor Meeting 6 - 01/11/2019}
This week I finished the sentence segmentation section from last week and started writing the machine translation section. Progress was a bit slower as I spent a lot of time reading to get a better understanding of the underlying theory behind NMT.

Dimitra suggested that I added information about using a transformer model for NMT as well as writing about translation evaluation techniques other than BLEU score (NIST and METEOR).

By the next meeting I am aiming to have completely finished the machine translation section and made a start on the machine learning section. This will likely be the most difficult section to write about in the literature review as it is very technical and there is a lot to cover.


\subsubsection{Supervisor Meeting 7 - 08/11/2019}
This week I completed the statistical machine translation section and added more technical information to the BLEU score translation evaluation section. I also started the machine learning section. This section was renamed to Deep Learning, to better reflect the content of the section. For this section so far, I have written two pages about CNNs.

Dimitra read the new content I had written for the literature review and found a few grammatical errors and places where I should add a citation to back up a statement. She also gave some very helpful feedback about what to include more information about for the remaining parts of the literature review so that I don’t spend too much time on subsections that aren’t important.

For next week I need to make the changes Dimitra recommend, finish the Deep Learning section (CNNs + RNNs), and add some more information about other methods of automatic evaluation. Once the other methods have been added, I need to compare them with the BLEU score metric.
